\chapter*{Závěr}
\addcontentsline{toc}{chapter}{Závěr}

Shrňme bakalářskou práci. Nejprve jsme si uvedli oblast zájmu, tj. trasování na Linuxu, kde jsme si ukázali, jak Linux podporuje trasování na systému a uvedli jsme i nějaké nástroje, které trasování umožňují (tracepointy), či zjednodušují (Trace-cmd). V následující kapitole jsme se podívali na vizualizaci trasování, zejména na některé nástroje, které vizualizují různá trasovací data, např. HPerf nebo Flame Graphs. Třetí kapitola popsala KernelShark, jeho strukturu a návod k použití. Tím jsme si omezili kontext pro software, který jsme popisovali v dalších kapitolách.

Ve čtvrté kapitole jsme analyzovali, jak bychom mohli KernelShark zlepšit, ať už skrze pluginy nebo modifikaci zdrojového kódu. Požadavky extrahované z této kapitoly jsme pak použili pro několik vylepšení, která jsme popsali v jejich samostatných kapitolách, včetně technické analýzy, vývojové a uživatelské dokumentace, kritiky a rozšíření. Vytvořili jsme pluginy Naps a Stacklook, první pomáhá vizualizovat nečinnost procesů a druhý dává uživateli příjemnější způsob prohlížení záznamů zásobníku kernelu. Krom pluginů jsme i modifikovali kód KernelSharku modifikacemi Record Kstack, Couplebreak a NUMA Topology Views. První modifikace byla lehká a dodala GUI od KernelSharku pro Trace-cmd jednoduchý ovládací prvek, přes který lze nastavit zaznamenávání zásobníků kernelu při trasování. Couplebreak naučil KernelShark rozdělovat některé události, čímž se stal zdrojem kompatibility pro pluginy, které by se jnak o události přely, například pluginy Naps a sched\_events. Poslední modifikace, NUMA Topology Views, přidala KernelSharku novou součást, se kterou může KernelShark zobrazovat vedle trasovacích dat i NUMA topologii systému. Vedle hlavních pluginů a modifikací jsme pak dodali i dodatečná vylepšení, která nám zjednodušila práci s programem, nebo opravovala méně důležité problémy KernelSharku, např. kód, který bude brzy nepodporován.