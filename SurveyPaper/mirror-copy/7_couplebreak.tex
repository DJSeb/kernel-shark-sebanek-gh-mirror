\chapter{Couplebreak}
V této kapitole se detailně seznámíme s modifikací pro KernelShark, která je schopna rozdělovat události a vytvářet pro ně záznamy KernelSharku. Tímto se zjemní dělení informací, což prospěje analýze a zároveň bude představovat formu kompatibility pro pluginy s jistými nároky. Představeny budou cíle, analýza řešení, návrh a použití tohoto vylepšení. Konečnou částí pak budou rozšíření, která by mohla tuto modifikaci dále zlepšit.

\section{Cíle}
\begin{itemize}
    \item Podporované události dvou procesů dají vzniknout dvěma záznamům - původci a cíli.
    \item Modifikace bude navržena rozšiřitelně o další události.
    \item Nové záznamy budou patřit tomu procesu z páru, který předtím událost nevlastnil.
    \item Nové záznamy budou obsahovat odkaz na záznam s původní událostí.
    \item Nové záznamy budou splňovat rozhraní dotazů na záznamy KernelSharku.
    \item Nové záznamy bude možné filtrovat jednoduchým filtrem.
    \item Vylepšení bude možno zapnout a vypnout. Toto nastavení bude možné uložit do relací KernelSharku a načíst je z nich.
    \item Součástí vylepšení bude i zajištění kompatibility s pluginem sched\_events.
    \item Podporovány budou alespoň události sched\_switch a sched\_waking.
\end{itemize}

\section{Terminologie}
\begin{itemize}
    \item \emph{Couple/pár} je označení pro dva procesy, které sdílí nějakou událost. Například trasovací událost sched\_waking, kdy nějaký proces rozhodne o probuzení jiného procesu, přirozeně obsahuje dva procesy - probouzejícího a probouzeného. Páry se dají často rozdělit na procesy cílové a počáteční.
    \item \emph{Couplebreak událost} je fiktivní událost vytvořená Couplebreakem. Couplebreak momentálně vytváří pouze události cílové. Modifikace toto vyznačuje v sufixu jména události jako \uv{[target]}. Navíc každá taková událost v KernelSharku obsahuje ve svém jméně prefix \uv{couplebreak/}, podobně jako události plánovače úloh obsahují prefix \uv{sched/}. Couplebreak události mají speciální ID začínající na hodnotě \texttt{-10 000}, které se postupně vždy o 1 snižuje.
    \item \emph{Couplebreak záznam} je záznam vytvářený Couplebreakem pro Couplebreak událost. Významnou vlastností těchto záznamů je, že odkazují na záznam s událostí, kvůli které byla Couplebreak událost vytvořena.
    \item \emph{Origin/počáteční proces} je proces z páru, pro který existuje nějaká událost, se kterou tento proces ovlivní druhý proces z páru.
    \item \emph{Origin event/počáteční událost} je označení pro událost, která náleží počátečnímu procesu.
    \item \emph{Target/cílový proces} je proces z páru, pro který existuje nějaká událost, která tento proces nějak ovlivní. 
    \item \emph{Target event/cílová událost} je označení pro událost, která náleží cílovému procesu.
    \item Termíny \emph{(datový) stream, záznam, událost, relace} jsou převzaty z terminologie KernelSharku.
\end{itemize}

\section{Analýza}
Cíle/úvod této sekce...

\section{Vývojová dokumentace}
Tato sekce hodlá předat čtenáři strukturu modifikace, podle které se lze v modifikaci orientovat. Zde bude představen seznam změněných souborů, tj. místa, kde se lze s kódem vylepšení setkat, a rozdělení modifikace do modulů, tj. abstraktnější dělení modifikace než pouze na funkce a atové struktury.

\subsection{Modifikované soubory}
Modifikace používá značku COUPLEBREAK v ohraničeních změn. Níže je abecedně seřazený seznam spolu s krátkým popiskem změn uvnitř souboru.
\begin{itemize}
    \item \emph{KsMainWindow.hpp/cpp} - v těchto souborech byly o hlavního okna přidány grafické elementy pro ovládání Couplebreaku přes GUI.
    \item \emph{KsWidgetsLib.hpp/cpp} - v těchto souborech byl definován nový widget, přes který se Couplebreak zapíná a vypíná pro jednotlivé streamy; zároveň se zde upravil widget filtrující události (třída \verb|KsEventCheckboxWidget|).
    \item \emph{KsUtils.hpp/cpp} - v těchto souborech byla definována pomocná C++ funkce k získání identifikátorů všech Couplebreak událostí aktivních ve streamu.
    \item \emph{libkshark.h/c} - v těchto souborech byla upravena datová struktura pro datové streamy, datová struktura definující rozhraní streamů, inicializace hodnot při alokaci a konstrukci nového streamu a nakonec byla přidána definice nové funkce v rozhraní streamů pro získání identifikátorů všech Couplebreak událostí aktivních ve streamu.
    \item \emph{libkshark-configio.c} - do tohoto souboru se přidalo ukládání stavu Couplebreaku do relací.
    \item \emph{libkshark-couplebreak.h/c} - v těchto souborech jsou definovány identifikátory pro jednotlivé Couplebreak události jako makra, pozice indikátorů v bitové masce aktivních Couplebreak událostí ve streamech a pomocné funkce s Couplebreakem spojené.
    \item \emph{libkshark-tepdata.c} - zde se Couplebreak záznamy vytvářejí a upravují; zároveň jsou zde upravené implementace rozhraní streamů a nastavování přidaných datových polí streamů.
    \item \emph{sched\_events.c} - zde byl plugin upraven tak, aby respektoval aktivní Couplebreak a aktivně ho využíval ve svůj prospěch.
\end{itemize}

\subsection{Struktura modifikace}

Modifikace rozdělíme na moduly. Nebudeme ovšem dělit jen dle funkcionalit. Modifikace obsahuje i několik částí, které ji integrují do existujících modulů KernelSharku. Jejich seskupením do \uv{integračního modulu} bychom přišli o mnoho informací a navíc ne každá integrace má stejné cíle jako jiná. Proto budou integrační části samostatnými moduly.

Couplebreak se dá rozdělit do pěti modulů: Couplebreak API, integrace s datovými streamy, spolupráce s relacemi, spolupráce sched\_events s Couplebreakem a konfigurace Couplebreaku. 

Couplebreak API...

Integrace s datovými streamy... proměnné, změny funkcí pro stream API

Spolupráce s relacemi...

Spolupráce sched\_events s Couplebreakem...

Konfigurace Couplebreaku...

\section{Uživatelská dokumentace}

\begin{code}
## GUI

All a GUI user has to do is navigate to the `Tools` menu and click on `Couplebreak Settings`, which should be just below
the `Record` button. A configuration window for each stream will pop up, along with an explanation of this modification.

Checking the checkboxes for which couplebreak should activate and applying them will have the affected streams insert new
couplebreak events, which can be filtered, selected, found and interacted with.

Couplebreak entries can be filtered by the simple checkbox filter as any other event, advanced filtering is currently
NOT supported. The entries will be visible in the main window's list view and in the graph. They might overlap with
their origin events, but even if that happens, their nature as a "twin" of the origin event makes this a non-issue.

## API

### Couplebreak API

Check for couplebreak events via the provided couplebreak event Id macros. Check the flag bitmask for present
couplebreak event types in a stream with the flag position macros. Query for flag position, event Id, event name or
origin event via the provided functions.

To detect a couplebreak event, look for entries with event Ids same as any of the "COUPLEBREAK_\[EVENT ABBREVIATION\]T_ID"
macros, where \[EVENT ABBREVIATION\] is something like SW, which symbolises `sched/sched_waking`. The T symbolises target.

By including the couplebreak header file, your code agrees to respect the possibility of being used when couplebreak is
active.

### Data streams

Streams' new state variables `couplebreak_on`, `n_couplebreak_evts` and `couplebreak_evts_flags` are now present. Use
`n_couplebreak_evts` to check how many couplebreak events there are in a stream. The flags bitmask sets each bit to 1
if a couplebreak event for a type of an event was created. Positions are 0-indexed (0th bit is least significant bit) and 
each bit flags an event as present as follows:
- 0 -> `couplebreak/sched_switch[target]` (event ID: `COUPLEBREAK_SST_ID`)
- 1 -> `couplebreak/sched_waking[target]` (event ID: `COUPLEBREAK_SWT_ID`)
The bitmask also serves as list of supported events.

It is recommended to not touch these, same as not touching `n_events` in streams, as this modification depends on them 
heavily.

Every function of a stream interface can be used in the same manner as for any other events, but couplebreak events will
most likely either redirect the request to their origin Id or construct values on the fly, they will not read the trace
data source.

### Couplebreak event entries

Couplebreak events are dynamically added and have no way of checking their "original data", except redirecting requests
for those to their origin entry or reconstructing that data as if being made anew, who must have been found in the data 
source for the current trace graph. Consequences of this are as follows:
- Couplebreak events **MUST NEVER** have their `event_id` and `offset` fields changed.
  - Offset has been changed to the origin entry's address - everything will break if this is changed.
  - Event Id is created on data load and cannot be found later somewhere else - asking for the original will just
    redirect to `entry->event_id`. Changing a couplebreak entry's event Id is the same as breaking it.

The entries can be used in plugins (e.g. `sched_event`'s couplebreak integration) and they shine especially well when
something needs to have an event in one task plot, but it cannot be done without editing an entry, possibly creating
an incompatiblity with other plugins. Using couplebreak events can resolve this. 

# Bugs

There have been two occurences of possibly parallel execution of KernelShark causing a crash when correcting CPUs of
`sched/sched_waking` events. These were only encountered during session support development. As such, they may have already
been resolved. However, because of their unexpected appearences, they must be at least mentioned. If one encounters such
an issue, it is recommended to simply restart the program and try again.
\end{code}

\subsection{Uživatel GUI}

\subsection{Vývojář pluginů}

\subsection{Vývojář KernelSharku}

\section{Rozšíření}

Více událostí - včetně vytváření \uv{[origin]} událostí...

Optimalizace průchodů daty při opravě CPU pro sched\_waking...

Méně Couplebreak proměnných v kshark\_data\_stream...

get\_all\_event\_ids vs get\_couplebreak\_evt\_ids...

Změny funkcí ve stream API a oddělení od ostatních custom events jako missed\_events...