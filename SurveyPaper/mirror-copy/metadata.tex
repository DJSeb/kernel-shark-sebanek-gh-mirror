%%% Vyplňte prosím základní údaje o závěrečné práci (odstraňte \xxx{...}).
%%% Automaticky se pak vloží na všechna místa, kde jsou potřeba.

% Druh práce:
%	"bc" pro bakalářskou
%	"mgr" pro diplomovou
%	"phd" pro disertační
%	"rig" pro rigorozní
\def\ThesisType{bc}

% Název práce v jazyce práce (přesně podle zadání)
\def\ThesisTitle{{Vizualizace trasování procesů v Linuxu}}

% Název práce v angličtině
\def\ThesisTitleEN{{Visualization of tracing of processes in Linux}}

% Jméno autora (vy)
\def\ThesisAuthor{{David Jaromír Šebánek}}

% Rok odevzdání
\def\YearSubmitted{{2025}}

% Název katedry nebo ústavu, kde byla práce oficiálně zadána
% (dle Organizační struktury MFF UK:
% https://www.mff.cuni.cz/cs/fakulta/organizacni-struktura,
% případně plný název pracoviště mimo MFF)
\def\Department{{Katedra distribuovaných a spolehlivých systémů}}
\def\DepartmentEN{{Department of Distributed and Dependable Systems}}

% Jedná se o katedru (department) nebo o ústav (institute)?
\def\DeptType{{Katedra}}
\def\DeptTypeEN{{Department}}

% Vedoucí práce: Jméno a příjmení s~tituly
\def\Supervisor{{RNDr. Jan Kára, Ph.D.}}

% Pracoviště vedoucího (opět dle Organizační struktury MFF)
\def\SupervisorsDepartment{{Katedra distribuovaných a spolehlivých systémů}}
\def\SupervisorsDepartmentEN{{Department of Distributed and Dependable Systems}}

% Studijní program (kromě rigorozních prací)
\def\StudyProgramme{{Informatika (B0613A140006)}}

% Nepovinné poděkování (vedoucímu práce, konzultantovi, tomu, kdo
% vám nosil pizzu a vařil čaj apod.)
\def\Dedication{%
\xxx{Poděkování.}
    iLoveImg upscaling, VS Code, ChatGPT, Qt dokumentace, hwloc dokumentace, Linux kernel dokumentace, WSL, Yordan K., J. Kára, atd.
}

% Abstrakt (doporučený rozsah cca 80-200 slov; nejedná se o zadání práce)
\def\Abstract{%
\xxx{Abstrakt.}
}

% Anglická verze abstraktu
\def\AbstractEN{%
\xxx{Abstract.}
}

% 3 až 5 klíčových slov (doporučeno) oddělených \sep
% Hodí se pro nalezení práce podle tématu.
\def\ThesisKeywords{%
{Linux\sep trasování}
}

\def\ThesisKeywordsEN{
{Linux\sep tracing}
}

% Pokud některá z položek metadat obsahuje TeXové řídící sekvence, je potřeba
% dodat i verzi v obyčejném textu, která se objeví v metadatech formátu XMP
% zabudovaných do výstupního souboru PDF. Pokud si nejste jistí, prohlédněte si
% vygenerovaný soubor thesis.xmpdata.
\def\ThesisAuthorXMP{\ThesisAuthor}
\def\ThesisTitleXMP{\ThesisTitle}
\def\ThesisKeywordsXMP{\ThesisKeywords}
\def\AbstractXMP{\Abstract}

% Máte-li dlouhý abstrakt a nechceme se mu vejít na informační stranu,
% můžete použít toto nastavení ke zmenšení písma informační strany.
% (Uvažte nicméně zkrácení abstraktu, to je často lepší.)
\def\InfoPageFont{}
%\def\InfoPageFont{\small}  % odkomentujte pro zmenšení písma
