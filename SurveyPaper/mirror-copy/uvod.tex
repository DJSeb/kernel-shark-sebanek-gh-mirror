\chapter*{Úvod}
\addcontentsline{toc}{chapter}{Úvod}

Na světě je mnoho počítačů, mnoho operačních systémů a mnohem více programů pro tyto systémy. Moderní systémy hojně využívají přepínání mezi programy k navýšení výkonu, snaží se tak být co nejefektivnější. Mnoho větších systémů je dnes distribuovaných, s mnoha procesory a technologiemi, které se snaží využít tyto výpočetní jednotky v maximální podobě.

Ovšem systém není vždy schopen sám zvýšit výkon. V tom případě je nutné začít zkoumat, kde se práce zdržuje, na co se nejvíce čeká a proč se na to čeká. Nalezený problém se pak může hlouběji zanalyzovat a výkon tak s vyřešeným problémem navýšit.

Na hledání problémů existuje mnoho metod, nicméně tato práce se bude zabývat pouze jednou z nich - trasování systému. Trasování systému je ve zkratce úkon, při kterém se na systému spustí nějaká práce společně s programem, který zaznamenává, co přesně systém během práce dělá. Program pak zaznamenaná data uloží. Tato data jsou ovšem často zakódována tak, aby se šetřilo místem - uživatel z těchto dat mnohem více spíš nic nevyčte než naopak.

Na záchranu přicházejí interpreti těchto dat, často s grafickým prostředím, jejich cílem je vizualizace trasování. Vizualizace pak představí data uživateli v takovém formátu, že v nich již lze hledat místa, kde výkon systému nebyl dostatečně vysoký. Jedním z takových vizualizačních nástrojů je program KernelShark pro Linux. Avšak žádný program není dokonalý, KernelShark nevyjímaje. A ani trasování nedokáže zachytit všechny informace o systému.

Jedním z příkladů toho, co KernelShark nesvede je zobrazení informace, z jakého stavu byl proces přepnut preemptivním přepnutím kontextu. Další nedokonalost se objevuje při zkoumání zásobníku kernelu Linuxu. KernelShark dokáže zásobník zobrazit, ale toto zobrazení není moc uživatelsky příjemné. Nedostatky můžeme najít i u trasovacích dat samotných. Existují události, které jsou podstatné pro dva procesy, ale pouze jeden smí tuto událost vlastnit, druhý proces o události ani neví. Analýza problémů pak musí spoléhat na dalších několik nástrojů pro získání celé představy o systému. Uživatel tak ale musí interagovat s vícero programy naráz. To ale představuje dodatečnou práci k analýze.

\section*{Cíle práce}
Cíle práce jsou primárně dva: hlouběji představit trasování, jeho vizualizaci a KernelShark, než jak je pouze nastiňuje úvod, a vylepšit KernelShark tak, aby analýza trasovacích dat byla informativnější a užvatelsky příjemnější.

\section*{Struktura práce}
Kapitoly práce lze rozdělit na poloviny. První polovina je teoretická a její součástí jsou kapitoly jedna až tři. V nich se hlouběji popisuje trasování v Linuxu, nástroje pro sběr a vizualizaci trasovacích dat a speciálně věnuje jednu kapitolu KernelSharku. Druhá polovina je zaměřená na vylepšení KernelSharku a pokrývá kapitoly čtyři až deset. Zde se analyzují požadavky na vylepšení, vytvářejí se technická rozhodnutí pro implementaci, součástí jsou i vývojové a uživatelské dokumentace, spolu s rozšířeními pro každé vylepšení a příklady využití. Každé vylepšení má vlastní kapitolu, dodatečná vylepšení jsou seskupena v jedné kapitole.

Závěr práci shrnuje a zhodnocuje splnění požadavků.