\chapter*{Úvod}
\addcontentsline{toc}{chapter}{Úvod}

Na světě je mnoho počítačů, mnoho operačních systémů a mnohem více programů pro tyto systémy. Moderní systémy hojně využívají přepínání mezi programy k zefektivnění využívání existujících zdrojů, snaží se tak být co nejefektivnější. Mnoho větších systémů je dnes distribuovaných, s mnoha procesory a technologiemi, které se snaží využít tyto výpočetní jednotky v maximální možné míře.

Ovšem systém není vždy schopen sám zvýšit výkon. V tom případě je nutné začít zkoumat, kde se práce zdržuje, na co se nejvíce čeká a proč se na to čeká. Nalezený problém se pak může hlouběji zanalyzovat a výkon tak s vyřešeným problémem navýšit.

Na hledání problémů existuje mnoho metod, nicméně tato práce se bude zabývat pouze jednou z nich - trasování systému. Trasování systému je ve zkratce úkon, při kterém se na systému spustí nějaká práce společně s programem, který zaznamenává, co přesně systém během práce dělá. Program pak zaznamenaná data uloží. Tato data jsou ovšem často zakódována tak, aby se šetřilo místem - uživatel z těchto dat mnohem více spíš nic nevyčte než naopak.

Na záchranu přicházejí interpreti těchto dat, často s grafickým prostředím, jejichž cílem je vizualizace trasování. Vizualizace pak představí data uživateli v takovém formátu, že v nich již lze hledat místa, kde výkon systému nebyl dostatečně vysoký. Jedním z takových vizualizačních nástrojů je program KernelShark pro Linux. Avšak žádný program není dokonalý, KernelShark nevyjímaje.

KernelShark dokáže efektivně zobrazit rozhodnutí systémového plánovače úloh, na jakém CPU proces pracoval, než šel spát, na jakém CPU práci obnoví, jak dlouho období nečinnosti trvá a podobně. KernelShark ale neumožňuje snadno získat důvod uspání procesu, na kterou událost nebo proces čeká. K tomuto je nutné využít další nástroje. Analýza problémů pak musí spoléhat na dalších několik nástrojů pro získání celé představy o systému a událostech v něm, což je v praxi obtížné, často až nemožné.

\section*{Cíle práce}
Cíle práce jsou primárně dva: hlouběji představit trasování, jeho vizualizaci a KernelShark, než jak je pouze nastiňuje úvod, a vylepšit KernelShark tak, aby analýza trasovacích dat byla informativnější a uživatelsky příjemnější.

\section*{Struktura práce}
Kapitoly práce lze rozdělit na dvě části. První část je teoretická a její součástí jsou kapitoly jedna až tři. V nich se hlouběji popisuje trasování v Linuxu, nástroje pro sběr a vizualizaci trasovacích dat a speciálně věnuje jednu kapitolu KernelSharku. Druhá část je zaměřená na vylepšení KernelSharku a pokrývá kapitoly čtyři až deset. Zde se analyzují požadavky na vylepšení, vytvářejí se technická rozhodnutí pro implementaci, součástí jsou i vývojové a uživatelské dokumentace, spolu s rozšířeními pro každé vylepšení, příklady využití a zhodnocení splnění podmínek. Každé vylepšení má vlastní kapitolu, dodatečná vylepšení jsou seskupena v jedné kapitole a jejich popisy jsou stručnější. Závěr práci shrnuje.