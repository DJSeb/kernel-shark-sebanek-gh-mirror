\chapter{NUMA Topology Views}
Cíle/úvod této kapitoly...

\section{Cíle}

\begin{itemize}
    \item Modifikace bude umět zpracovat topologická data z XML souboru vytvořeného programem hwloc. Z tohoto souboru bude hlavně chtít vyčíst NUMA topologii procesorů.
    \item Zpracovaná topologická data budou zobrazena někde v hlavním okně. Místo zobrazení by mělo dovolovat přirozenou návaznost na CPU grafy. Ty mohou být přeuspořádány tak, aby respektovaly řazení v topologii.
    \item Pokud nemáme topologická data k dispozici pro streamy, nebudeme topologii pro dané streamy zobrazovat.
    \item Topologie budou zobrazovány jako stromy.
    \item Každý prvek stromu bude viditelně pojmenován. Pokud by jméno bylo příliš dlouhé, lze použít popisky při najetí myši.
    \item Topologie nebudou zobrazovat NUMA uzly, pokud existuje pouze jeden (a NUMA technologie je tedy nevyužitá).
    \item Topologie budou vždy vykreslovat alespoň jádra v topologii. Ta budou vždy obsahovat alespoň jeden procesor.
    \item Jádra budou zabarvena průměrnou barvou ze svých procesů. NUMA uzly budou zabarveny průměrnou barvou jader, která jsou součástí uzlu.
    \item Místo s topologickými stromy bude možné schovat přes GUI prvek.
    \item Modifikace bude mít konfigurační okénko, ve kterém si bude uživatel schopen vybrat soubor s topologickými daty a typ pohledu na stream - buď klasický, nebo se zobrazením topologie. Pokud nebude vybrána topologie, ale bude vybrán topologický pohled, bude namísto toho použit klasický pohled. Vybrání souboru topologie s odlišným počtem CPU, než jsou v daném streamu tuto topologii nezobrazí, použije klasický pohled a uživatele o nesrovnalosti informuje.
    \item Modifikace bude uložitelná do relací.
\end{itemize}


\section{Analýza}
Cíle/úvod této sekce...


\subsection{Terminologie}


\section{Vývojová dokumentace}
Cíle/úvod této sekce...

\subsection{Modifikované soubory}

\subsection{Struktura modifikace}

\section{Uživatelská dokumentace}
Cíle/úvod této sekce...

\subsection{Nové závislosti KernelSharku}

\subsection{Uživatel GUI}

\subsection{Vývojář pluginů}

\subsection{Vývojář KernelSharku}

\section{Rozšíření}
Cíle/úvod této sekce...

\section{Zhodnocení splněných požadavků}